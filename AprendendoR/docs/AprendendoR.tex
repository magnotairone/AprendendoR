% Options for packages loaded elsewhere
\PassOptionsToPackage{unicode}{hyperref}
\PassOptionsToPackage{hyphens}{url}
%
\documentclass[
]{book}
\usepackage{amsmath,amssymb}
\usepackage{iftex}
\ifPDFTeX
  \usepackage[T1]{fontenc}
  \usepackage[utf8]{inputenc}
  \usepackage{textcomp} % provide euro and other symbols
\else % if luatex or xetex
  \usepackage{unicode-math} % this also loads fontspec
  \defaultfontfeatures{Scale=MatchLowercase}
  \defaultfontfeatures[\rmfamily]{Ligatures=TeX,Scale=1}
\fi
\usepackage{lmodern}
\ifPDFTeX\else
  % xetex/luatex font selection
\fi
% Use upquote if available, for straight quotes in verbatim environments
\IfFileExists{upquote.sty}{\usepackage{upquote}}{}
\IfFileExists{microtype.sty}{% use microtype if available
  \usepackage[]{microtype}
  \UseMicrotypeSet[protrusion]{basicmath} % disable protrusion for tt fonts
}{}
\makeatletter
\@ifundefined{KOMAClassName}{% if non-KOMA class
  \IfFileExists{parskip.sty}{%
    \usepackage{parskip}
  }{% else
    \setlength{\parindent}{0pt}
    \setlength{\parskip}{6pt plus 2pt minus 1pt}}
}{% if KOMA class
  \KOMAoptions{parskip=half}}
\makeatother
\usepackage{xcolor}
\usepackage{color}
\usepackage{fancyvrb}
\newcommand{\VerbBar}{|}
\newcommand{\VERB}{\Verb[commandchars=\\\{\}]}
\DefineVerbatimEnvironment{Highlighting}{Verbatim}{commandchars=\\\{\}}
% Add ',fontsize=\small' for more characters per line
\usepackage{framed}
\definecolor{shadecolor}{RGB}{248,248,248}
\newenvironment{Shaded}{\begin{snugshade}}{\end{snugshade}}
\newcommand{\AlertTok}[1]{\textcolor[rgb]{0.94,0.16,0.16}{#1}}
\newcommand{\AnnotationTok}[1]{\textcolor[rgb]{0.56,0.35,0.01}{\textbf{\textit{#1}}}}
\newcommand{\AttributeTok}[1]{\textcolor[rgb]{0.13,0.29,0.53}{#1}}
\newcommand{\BaseNTok}[1]{\textcolor[rgb]{0.00,0.00,0.81}{#1}}
\newcommand{\BuiltInTok}[1]{#1}
\newcommand{\CharTok}[1]{\textcolor[rgb]{0.31,0.60,0.02}{#1}}
\newcommand{\CommentTok}[1]{\textcolor[rgb]{0.56,0.35,0.01}{\textit{#1}}}
\newcommand{\CommentVarTok}[1]{\textcolor[rgb]{0.56,0.35,0.01}{\textbf{\textit{#1}}}}
\newcommand{\ConstantTok}[1]{\textcolor[rgb]{0.56,0.35,0.01}{#1}}
\newcommand{\ControlFlowTok}[1]{\textcolor[rgb]{0.13,0.29,0.53}{\textbf{#1}}}
\newcommand{\DataTypeTok}[1]{\textcolor[rgb]{0.13,0.29,0.53}{#1}}
\newcommand{\DecValTok}[1]{\textcolor[rgb]{0.00,0.00,0.81}{#1}}
\newcommand{\DocumentationTok}[1]{\textcolor[rgb]{0.56,0.35,0.01}{\textbf{\textit{#1}}}}
\newcommand{\ErrorTok}[1]{\textcolor[rgb]{0.64,0.00,0.00}{\textbf{#1}}}
\newcommand{\ExtensionTok}[1]{#1}
\newcommand{\FloatTok}[1]{\textcolor[rgb]{0.00,0.00,0.81}{#1}}
\newcommand{\FunctionTok}[1]{\textcolor[rgb]{0.13,0.29,0.53}{\textbf{#1}}}
\newcommand{\ImportTok}[1]{#1}
\newcommand{\InformationTok}[1]{\textcolor[rgb]{0.56,0.35,0.01}{\textbf{\textit{#1}}}}
\newcommand{\KeywordTok}[1]{\textcolor[rgb]{0.13,0.29,0.53}{\textbf{#1}}}
\newcommand{\NormalTok}[1]{#1}
\newcommand{\OperatorTok}[1]{\textcolor[rgb]{0.81,0.36,0.00}{\textbf{#1}}}
\newcommand{\OtherTok}[1]{\textcolor[rgb]{0.56,0.35,0.01}{#1}}
\newcommand{\PreprocessorTok}[1]{\textcolor[rgb]{0.56,0.35,0.01}{\textit{#1}}}
\newcommand{\RegionMarkerTok}[1]{#1}
\newcommand{\SpecialCharTok}[1]{\textcolor[rgb]{0.81,0.36,0.00}{\textbf{#1}}}
\newcommand{\SpecialStringTok}[1]{\textcolor[rgb]{0.31,0.60,0.02}{#1}}
\newcommand{\StringTok}[1]{\textcolor[rgb]{0.31,0.60,0.02}{#1}}
\newcommand{\VariableTok}[1]{\textcolor[rgb]{0.00,0.00,0.00}{#1}}
\newcommand{\VerbatimStringTok}[1]{\textcolor[rgb]{0.31,0.60,0.02}{#1}}
\newcommand{\WarningTok}[1]{\textcolor[rgb]{0.56,0.35,0.01}{\textbf{\textit{#1}}}}
\usepackage{longtable,booktabs,array}
\usepackage{calc} % for calculating minipage widths
% Correct order of tables after \paragraph or \subparagraph
\usepackage{etoolbox}
\makeatletter
\patchcmd\longtable{\par}{\if@noskipsec\mbox{}\fi\par}{}{}
\makeatother
% Allow footnotes in longtable head/foot
\IfFileExists{footnotehyper.sty}{\usepackage{footnotehyper}}{\usepackage{footnote}}
\makesavenoteenv{longtable}
\usepackage{graphicx}
\makeatletter
\def\maxwidth{\ifdim\Gin@nat@width>\linewidth\linewidth\else\Gin@nat@width\fi}
\def\maxheight{\ifdim\Gin@nat@height>\textheight\textheight\else\Gin@nat@height\fi}
\makeatother
% Scale images if necessary, so that they will not overflow the page
% margins by default, and it is still possible to overwrite the defaults
% using explicit options in \includegraphics[width, height, ...]{}
\setkeys{Gin}{width=\maxwidth,height=\maxheight,keepaspectratio}
% Set default figure placement to htbp
\makeatletter
\def\fps@figure{htbp}
\makeatother
\setlength{\emergencystretch}{3em} % prevent overfull lines
\providecommand{\tightlist}{%
  \setlength{\itemsep}{0pt}\setlength{\parskip}{0pt}}
\setcounter{secnumdepth}{5}
\usepackage{booktabs}
\ifLuaTeX
  \usepackage{selnolig}  % disable illegal ligatures
\fi
\usepackage[]{natbib}
\bibliographystyle{plainnat}
\IfFileExists{bookmark.sty}{\usepackage{bookmark}}{\usepackage{hyperref}}
\IfFileExists{xurl.sty}{\usepackage{xurl}}{} % add URL line breaks if available
\urlstyle{same}
\hypersetup{
  pdftitle={Aprendendo R},
  pdfauthor={Magno TF Severino},
  hidelinks,
  pdfcreator={LaTeX via pandoc}}

\title{Aprendendo R}
\author{Magno TF Severino}
\date{2023-12-06}

\begin{document}
\maketitle

{
\setcounter{tocdepth}{1}
\tableofcontents
}
\chapter{Intro}\label{intro}

Este livro compila tutoriais da linguagem \texttt{R} que usei em diversas aulas que dei ao longo dos últimos anos.

\chapter{\texorpdfstring{Básicos da linguagem \texttt{R}}{Básicos da linguagem R}}\label{buxe1sicos-da-linguagem-r}

Para instalação,

\begin{itemize}
\item
  faça o download do R em \url{http://www.r-project.org};
\item
  sugestão: utilizar a IDE \href{http://www.rstudio.org}{R Studio}.
\end{itemize}

É muito importante saber como obter ajuda no R.
Sempre que estiver em dúvidas quanto as características de alguma função, consule a aba \emph{help} do R.

\begin{Shaded}
\begin{Highlighting}[]
\NormalTok{?mean }\CommentTok{\#abre a página de ajuda da função \textquotesingle{}mean\textquotesingle{}}
\NormalTok{??plot }\CommentTok{\#procura por tópicos contendo a palavra \textquotesingle{}plot\textquotesingle{}}
\end{Highlighting}
\end{Shaded}

\section{\texorpdfstring{Usando o \texttt{R} como uma calculadora}{Usando o R como uma calculadora}}\label{usando-o-r-como-uma-calculadora}

O operador + realiza a adição entre dois elementos.

\begin{Shaded}
\begin{Highlighting}[]
\DecValTok{1} \SpecialCharTok{+} \DecValTok{2}
\end{Highlighting}
\end{Shaded}

\begin{verbatim}
## [1] 3
\end{verbatim}

Um vetor é um conjunto ordenado de valores.
O operador : cria uma sequência a partir de um número até outro.
A função c concatena valores, criando um vetor.

\begin{Shaded}
\begin{Highlighting}[]
\DecValTok{1}\SpecialCharTok{:}\DecValTok{5}
\end{Highlighting}
\end{Shaded}

\begin{verbatim}
## [1] 1 2 3 4 5
\end{verbatim}

\begin{Shaded}
\begin{Highlighting}[]
\FunctionTok{c}\NormalTok{(}\DecValTok{1}\NormalTok{, }\DecValTok{2}\NormalTok{, }\DecValTok{3}\NormalTok{, }\DecValTok{4}\NormalTok{, }\DecValTok{5}\NormalTok{)}
\end{Highlighting}
\end{Shaded}

\begin{verbatim}
## [1] 1 2 3 4 5
\end{verbatim}

Além de adicionar dois números, o operador + pode ser usado para adicionar dois vetores.

\begin{Shaded}
\begin{Highlighting}[]
\DecValTok{1}\SpecialCharTok{:}\DecValTok{5} \SpecialCharTok{+} \DecValTok{6}\SpecialCharTok{:}\DecValTok{10}
\end{Highlighting}
\end{Shaded}

\begin{verbatim}
## [1]  7  9 11 13 15
\end{verbatim}

\begin{Shaded}
\begin{Highlighting}[]
\FunctionTok{c}\NormalTok{(}\DecValTok{1}\NormalTok{, }\DecValTok{2}\NormalTok{, }\DecValTok{3}\NormalTok{, }\DecValTok{4}\NormalTok{, }\DecValTok{5}\NormalTok{) }\SpecialCharTok{+} \FunctionTok{c}\NormalTok{(}\DecValTok{6}\NormalTok{, }\DecValTok{7}\NormalTok{, }\DecValTok{8}\NormalTok{, }\DecValTok{9}\NormalTok{, }\DecValTok{10}\NormalTok{)}
\end{Highlighting}
\end{Shaded}

\begin{verbatim}
## [1]  7  9 11 13 15
\end{verbatim}

\begin{Shaded}
\begin{Highlighting}[]
\DecValTok{1}\SpecialCharTok{:}\DecValTok{5} \SpecialCharTok{+} \FunctionTok{c}\NormalTok{(}\DecValTok{6}\NormalTok{, }\DecValTok{7}\NormalTok{, }\DecValTok{8}\NormalTok{, }\DecValTok{9}\NormalTok{, }\DecValTok{10}\NormalTok{)}
\end{Highlighting}
\end{Shaded}

\begin{verbatim}
## [1]  7  9 11 13 15
\end{verbatim}

Os próximos exemplos mostram subtração, multiplicação, exponenciação e divisão.

\begin{Shaded}
\begin{Highlighting}[]
\FunctionTok{c}\NormalTok{(}\DecValTok{2}\NormalTok{, }\DecValTok{3}\NormalTok{, }\DecValTok{5}\NormalTok{, }\DecValTok{7}\NormalTok{, }\DecValTok{11}\NormalTok{, }\DecValTok{13}\NormalTok{) }\SpecialCharTok{{-}} \DecValTok{2}       \CommentTok{\#subtração}
\end{Highlighting}
\end{Shaded}

\begin{verbatim}
## [1]  0  1  3  5  9 11
\end{verbatim}

\begin{Shaded}
\begin{Highlighting}[]
\SpecialCharTok{{-}}\DecValTok{2}\SpecialCharTok{:}\DecValTok{2} \SpecialCharTok{*} \SpecialCharTok{{-}}\DecValTok{2}\SpecialCharTok{:}\DecValTok{2}                     \CommentTok{\#multiplicação}
\end{Highlighting}
\end{Shaded}

\begin{verbatim}
## [1] 4 1 0 1 4
\end{verbatim}

\begin{Shaded}
\begin{Highlighting}[]
\NormalTok{(}\DecValTok{1}\SpecialCharTok{:}\DecValTok{10}\NormalTok{) }\SpecialCharTok{\^{}} \DecValTok{2}                        \CommentTok{\#exponenciação}
\end{Highlighting}
\end{Shaded}

\begin{verbatim}
##  [1]   1   4   9  16  25  36  49  64  81 100
\end{verbatim}

\begin{Shaded}
\begin{Highlighting}[]
\DecValTok{1}\SpecialCharTok{:}\DecValTok{10} \SpecialCharTok{/} \DecValTok{3}                        \CommentTok{\#divisão}
\end{Highlighting}
\end{Shaded}

\begin{verbatim}
##  [1] 0.3333333 0.6666667 1.0000000 1.3333333 1.6666667 2.0000000 2.3333333
##  [8] 2.6666667 3.0000000 3.3333333
\end{verbatim}

\section{Atribuindo variáveis}\label{atribuindo-variuxe1veis}

Fazer cálculos com o R é bem simples e útil.
A maior parte das vezes queremos armazenar os resultados para uso posterior.
Assim, podemos atribuir valor à uma variável, através do operador \textless-.

\begin{Shaded}
\begin{Highlighting}[]
\NormalTok{a }\OtherTok{\textless{}{-}} \DecValTok{1}

\NormalTok{b }\OtherTok{\textless{}{-}} \DecValTok{5} \SpecialCharTok{*} \DecValTok{3}

\NormalTok{x }\OtherTok{\textless{}{-}} \DecValTok{1}\SpecialCharTok{:}\DecValTok{5}

\NormalTok{y }\OtherTok{\textless{}{-}} \DecValTok{6}\SpecialCharTok{:}\DecValTok{10}
\end{Highlighting}
\end{Shaded}

Agora, podemos reutilizar esses valores para fazer outros cálculos.

\begin{Shaded}
\begin{Highlighting}[]
\NormalTok{a }\SpecialCharTok{+} \DecValTok{2} \SpecialCharTok{*}\NormalTok{ b}
\end{Highlighting}
\end{Shaded}

\begin{verbatim}
## [1] 31
\end{verbatim}

\begin{Shaded}
\begin{Highlighting}[]
\NormalTok{x }\SpecialCharTok{+} \DecValTok{2} \SpecialCharTok{*}\NormalTok{ y }\SpecialCharTok{{-}} \DecValTok{3}
\end{Highlighting}
\end{Shaded}

\begin{verbatim}
## [1] 10 13 16 19 22
\end{verbatim}

Observe que não temos que dizer ao R qual o tipo da variável, se era um número (as variáveis a e a) ou vetor (x e y).

\section{Números especias}\label{nuxfameros-especias}

Para facilitar operações aritméticas, R suporta quatro valores especiais de números: Inf, -Inf, NaN e NA.
Os dois primeiros representam infinito positivo e negativo.
NaN é um acrônimo inglês para ``not a number'', ou seja, não é um número.
Ele aparece quando um cálculo não faz sentido, ou não está definido.
NA significa ``not available'', ou seja, não disponível, e representa um valor faltante.

\begin{Shaded}
\begin{Highlighting}[]
\FunctionTok{c}\NormalTok{(}\ConstantTok{Inf} \SpecialCharTok{+} \DecValTok{1}\NormalTok{, }\ConstantTok{Inf} \SpecialCharTok{{-}} \DecValTok{1}\NormalTok{, }\ConstantTok{Inf} \SpecialCharTok{{-}} \ConstantTok{Inf}\NormalTok{, }\ConstantTok{NA} \SpecialCharTok{+} \DecValTok{1}\NormalTok{)}
\end{Highlighting}
\end{Shaded}

\begin{verbatim}
## [1] Inf Inf NaN  NA
\end{verbatim}

\begin{Shaded}
\begin{Highlighting}[]
\FunctionTok{c}\NormalTok{(}\DecValTok{0} \SpecialCharTok{/} \DecValTok{0}\NormalTok{, }\ConstantTok{Inf} \SpecialCharTok{/} \ConstantTok{Inf}\NormalTok{, }\DecValTok{1} \SpecialCharTok{/} \ConstantTok{Inf}\NormalTok{)}
\end{Highlighting}
\end{Shaded}

\begin{verbatim}
## [1] NaN NaN   0
\end{verbatim}

\section{Vetores, Matrizes e Dataframes}\label{vetores-matrizes-e-dataframes}

Previamente, vimos alguns tipos de vetores para valores lógicos, caracteres e números.
Nessa seção, utilizaremos técnicas de manipulação de vetores e introduziremos o caso multidimensional: matrizes e dataframes.

Abaixo relembramos as operações que já foram feitas com vetores

\begin{Shaded}
\begin{Highlighting}[]
\DecValTok{10}\SpecialCharTok{:}\DecValTok{5}                    \CommentTok{\#sequência de números de 10 até 5}
\end{Highlighting}
\end{Shaded}

\begin{verbatim}
## [1] 10  9  8  7  6  5
\end{verbatim}

\begin{Shaded}
\begin{Highlighting}[]
\FunctionTok{c}\NormalTok{(}\DecValTok{1}\NormalTok{, }\DecValTok{2}\SpecialCharTok{:}\DecValTok{5}\NormalTok{, }\FunctionTok{c}\NormalTok{(}\DecValTok{6}\NormalTok{, }\DecValTok{7}\NormalTok{), }\DecValTok{8}\NormalTok{)   }\CommentTok{\#valores concatenados em um único vetor}
\end{Highlighting}
\end{Shaded}

\begin{verbatim}
## [1] 1 2 3 4 5 6 7 8
\end{verbatim}

\subsection{Vetores}\label{vetores}

Existem funções para criar vetores de um tipo e com tamanho específicos.
Todos os elementos deste vetor terá valor zero, FALSE, um caracter vazio, ou o equivalente à \emph{nada/vazio} para aquele tipo.
Veja abaixo duas maneiras de definir um vetor.

\begin{Shaded}
\begin{Highlighting}[]
\FunctionTok{vector}\NormalTok{(}\StringTok{"numeric"}\NormalTok{, }\DecValTok{5}\NormalTok{) }\CommentTok{\#cria um vetor numérico de 5 elementos}
\end{Highlighting}
\end{Shaded}

\begin{verbatim}
## [1] 0 0 0 0 0
\end{verbatim}

\begin{Shaded}
\begin{Highlighting}[]
\FunctionTok{numeric}\NormalTok{(}\DecValTok{5}\NormalTok{) }\CommentTok{\#equivalente ao comando acima}
\end{Highlighting}
\end{Shaded}

\begin{verbatim}
## [1] 0 0 0 0 0
\end{verbatim}

\begin{Shaded}
\begin{Highlighting}[]
\FunctionTok{vector}\NormalTok{(}\StringTok{"logical"}\NormalTok{, }\DecValTok{5}\NormalTok{) }\CommentTok{\#cria um vetor lógico de 5 elementos}
\end{Highlighting}
\end{Shaded}

\begin{verbatim}
## [1] FALSE FALSE FALSE FALSE FALSE
\end{verbatim}

\begin{Shaded}
\begin{Highlighting}[]
\FunctionTok{logical}\NormalTok{(}\DecValTok{5}\NormalTok{) }\CommentTok{\#equivalente ao comando acima}
\end{Highlighting}
\end{Shaded}

\begin{verbatim}
## [1] FALSE FALSE FALSE FALSE FALSE
\end{verbatim}

\begin{Shaded}
\begin{Highlighting}[]
\FunctionTok{vector}\NormalTok{(}\StringTok{"character"}\NormalTok{, }\DecValTok{5}\NormalTok{) }\CommentTok{\#cria um vetor de caracteres de 5 elementos}
\end{Highlighting}
\end{Shaded}

\begin{verbatim}
## [1] "" "" "" "" ""
\end{verbatim}

\begin{Shaded}
\begin{Highlighting}[]
\FunctionTok{character}\NormalTok{(}\DecValTok{5}\NormalTok{) }\CommentTok{\#equivalente ao comando acima}
\end{Highlighting}
\end{Shaded}

\begin{verbatim}
## [1] "" "" "" "" ""
\end{verbatim}

\subsubsection{Sequências}\label{S:sequencia}

Podemos criar sequências mais gerais que aquelas criadas com o operador :.
A função seq te permite criar sequências em diferentes maneiras
Veja abaixo.

\begin{Shaded}
\begin{Highlighting}[]
\FunctionTok{seq}\NormalTok{(}\DecValTok{3}\NormalTok{, }\DecValTok{12}\NormalTok{) }\CommentTok{\#equivalente à 3:12}
\end{Highlighting}
\end{Shaded}

\begin{verbatim}
##  [1]  3  4  5  6  7  8  9 10 11 12
\end{verbatim}

\begin{Shaded}
\begin{Highlighting}[]
\FunctionTok{seq}\NormalTok{(}\DecValTok{3}\NormalTok{, }\DecValTok{12}\NormalTok{, }\DecValTok{2}\NormalTok{) }\CommentTok{\#o terceiro argumento indica a distância entre os elementos na lista.}
\end{Highlighting}
\end{Shaded}

\begin{verbatim}
## [1]  3  5  7  9 11
\end{verbatim}

\begin{Shaded}
\begin{Highlighting}[]
\FunctionTok{seq}\NormalTok{(}\FloatTok{0.1}\NormalTok{, }\FloatTok{0.01}\NormalTok{, }\SpecialCharTok{{-}}\FloatTok{0.01}\NormalTok{)}
\end{Highlighting}
\end{Shaded}

\begin{verbatim}
##  [1] 0.10 0.09 0.08 0.07 0.06 0.05 0.04 0.03 0.02 0.01
\end{verbatim}

\subsubsection{Tamanhos}\label{tamanhos}

Todo vetor tem um tamanho, um número não negativo que representa a quantidade de elementos que o vetor contém.
A função length retorna o tamanho de um dado vetor.

\begin{Shaded}
\begin{Highlighting}[]
\FunctionTok{length}\NormalTok{(}\DecValTok{1}\SpecialCharTok{:}\DecValTok{5}\NormalTok{) }
\end{Highlighting}
\end{Shaded}

\begin{verbatim}
## [1] 5
\end{verbatim}

\begin{Shaded}
\begin{Highlighting}[]
\NormalTok{frase }\OtherTok{\textless{}{-}} \FunctionTok{c}\NormalTok{(}\StringTok{"Observe"}\NormalTok{, }\StringTok{"o"}\NormalTok{, }\StringTok{"resultado"}\NormalTok{, }\StringTok{"dos"}\NormalTok{, }\StringTok{"comandos"}\NormalTok{, }\StringTok{"abaixo"}\NormalTok{)}

\FunctionTok{length}\NormalTok{(frase)}
\end{Highlighting}
\end{Shaded}

\begin{verbatim}
## [1] 6
\end{verbatim}

\begin{Shaded}
\begin{Highlighting}[]
\FunctionTok{nchar}\NormalTok{(frase)}
\end{Highlighting}
\end{Shaded}

\begin{verbatim}
## [1] 7 1 9 3 8 6
\end{verbatim}

\subsubsection{Indexando vetores}\label{indexando-vetores}

A indexação é útil quando queremos acessar elementos específicos de um vetor.
Considere o vetor

\begin{Shaded}
\begin{Highlighting}[]
\NormalTok{x }\OtherTok{\textless{}{-}}\NormalTok{ (}\DecValTok{1}\SpecialCharTok{:}\DecValTok{5}\NormalTok{) }\SpecialCharTok{\^{}} \DecValTok{2}
\end{Highlighting}
\end{Shaded}

Abaixo, três métodos de indexar os mesmos valores do vetor x.

\begin{Shaded}
\begin{Highlighting}[]
\NormalTok{x[}\FunctionTok{c}\NormalTok{(}\DecValTok{1}\NormalTok{, }\DecValTok{3}\NormalTok{, }\DecValTok{5}\NormalTok{)]}

\NormalTok{x[}\FunctionTok{c}\NormalTok{(}\SpecialCharTok{{-}}\DecValTok{2}\NormalTok{, }\SpecialCharTok{{-}}\DecValTok{4}\NormalTok{)]}
\end{Highlighting}
\end{Shaded}

\begin{Shaded}
\begin{Highlighting}[]
\NormalTok{x[}\FunctionTok{c}\NormalTok{(}\ConstantTok{TRUE}\NormalTok{, }\ConstantTok{FALSE}\NormalTok{, }\ConstantTok{TRUE}\NormalTok{, }\ConstantTok{FALSE}\NormalTok{, }\ConstantTok{TRUE}\NormalTok{)]}
\end{Highlighting}
\end{Shaded}

\begin{verbatim}
## [1]  1  9 25
\end{verbatim}

Se nomearmos os elementos do vetor, o método abaixo obtém os mesmos valores de x.

\begin{Shaded}
\begin{Highlighting}[]
\FunctionTok{names}\NormalTok{(x) }\OtherTok{\textless{}{-}} \FunctionTok{c}\NormalTok{(}\StringTok{"one"}\NormalTok{, }\StringTok{"four"}\NormalTok{, }\StringTok{"nine"}\NormalTok{, }\StringTok{"sixteen"}\NormalTok{, }\StringTok{"twenty five"}\NormalTok{)}
\NormalTok{x[}\FunctionTok{c}\NormalTok{(}\StringTok{"one"}\NormalTok{, }\StringTok{"nine"}\NormalTok{, }\StringTok{"twenty five"}\NormalTok{)]}
\end{Highlighting}
\end{Shaded}

\begin{verbatim}
##         one        nine twenty five 
##           1           9          25
\end{verbatim}

Cuidado, acessar um elemento fora do tamanho do vetor não gera um erro no R, apenas NA.

\begin{Shaded}
\begin{Highlighting}[]
\NormalTok{x[}\DecValTok{6}\NormalTok{]}
\end{Highlighting}
\end{Shaded}

\begin{verbatim}
## <NA> 
##   NA
\end{verbatim}

\subsection{Matrizes}\label{matrizes}

Uma matriz é o equivalente à um vetor, porém em duas dimensões.
Abaixo, um exemplo de definição de uma matriz com 4 linhas e 3 colunas (total de 12 elementos).

\begin{Shaded}
\begin{Highlighting}[]
\NormalTok{?matrix}
\end{Highlighting}
\end{Shaded}

\begin{Shaded}
\begin{Highlighting}[]
\NormalTok{uma\_matriz }\OtherTok{\textless{}{-}} \FunctionTok{matrix}\NormalTok{(}
      \DecValTok{1}\SpecialCharTok{:}\DecValTok{12}\NormalTok{,}
      \AttributeTok{nrow =} \DecValTok{4}\NormalTok{, }\CommentTok{\#ncol = 3 gera o mesmo resultado. Verifique!}
      \AttributeTok{dimnames =} \FunctionTok{list}\NormalTok{(}
        \FunctionTok{c}\NormalTok{(}\StringTok{"L1"}\NormalTok{, }\StringTok{"L2"}\NormalTok{, }\StringTok{"L3"}\NormalTok{, }\StringTok{"L4"}\NormalTok{),}
        \FunctionTok{c}\NormalTok{(}\StringTok{"C1"}\NormalTok{, }\StringTok{"C2"}\NormalTok{, }\StringTok{"C3"}\NormalTok{)}
\NormalTok{      )}
\NormalTok{)}

\FunctionTok{class}\NormalTok{(uma\_matriz)}
\end{Highlighting}
\end{Shaded}

\begin{verbatim}
## [1] "matrix" "array"
\end{verbatim}

\begin{Shaded}
\begin{Highlighting}[]
\NormalTok{uma\_matriz}
\end{Highlighting}
\end{Shaded}

\begin{verbatim}
##    C1 C2 C3
## L1  1  5  9
## L2  2  6 10
## L3  3  7 11
## L4  4  8 12
\end{verbatim}

Por padrão, ao criar uma matrix, o vetor passado como primeiro argumento preenche a matrix por colunas.
Para preencher a matrix por linhas, basta especificar o argumento byrow=TRUE

A função dim retorna um vetor de inteiros com as dimensões da variável.

\begin{Shaded}
\begin{Highlighting}[]
\FunctionTok{dim}\NormalTok{(uma\_matriz)}
\end{Highlighting}
\end{Shaded}

\begin{verbatim}
## [1] 4 3
\end{verbatim}

\begin{Shaded}
\begin{Highlighting}[]
\FunctionTok{nrow}\NormalTok{(uma\_matriz) }\CommentTok{\#retorna o número de linhas da matrix}
\end{Highlighting}
\end{Shaded}

\begin{verbatim}
## [1] 4
\end{verbatim}

\begin{Shaded}
\begin{Highlighting}[]
\FunctionTok{ncol}\NormalTok{(uma\_matriz) }\CommentTok{\#retorna o número de colunas da matrix}
\end{Highlighting}
\end{Shaded}

\begin{verbatim}
## [1] 3
\end{verbatim}

\begin{Shaded}
\begin{Highlighting}[]
\FunctionTok{length}\NormalTok{(uma\_matriz) }\CommentTok{\#retorna o número de elementos da matrix}
\end{Highlighting}
\end{Shaded}

\begin{verbatim}
## [1] 12
\end{verbatim}

\subsubsection{Nomeando linhas, columas e dimensões}\label{nomeando-linhas-columas-e-dimensuxf5es}

Da mesma forma para vetores, podemos nomear (e obter os nomes de) linhas e colunas de matrizes.

\begin{Shaded}
\begin{Highlighting}[]
\FunctionTok{rownames}\NormalTok{(uma\_matriz)}
\end{Highlighting}
\end{Shaded}

\begin{verbatim}
## [1] "L1" "L2" "L3" "L4"
\end{verbatim}

\begin{Shaded}
\begin{Highlighting}[]
\FunctionTok{colnames}\NormalTok{(uma\_matriz)}
\end{Highlighting}
\end{Shaded}

\begin{verbatim}
## [1] "C1" "C2" "C3"
\end{verbatim}

\begin{Shaded}
\begin{Highlighting}[]
\FunctionTok{dimnames}\NormalTok{(uma\_matriz)}
\end{Highlighting}
\end{Shaded}

\begin{verbatim}
## [[1]]
## [1] "L1" "L2" "L3" "L4"
## 
## [[2]]
## [1] "C1" "C2" "C3"
\end{verbatim}

\subsubsection{Indexação}\label{indexauxe7uxe3o}

A indexação de matrizes funciona de maneira similar à de vetores, com a diferença que agora precisam ser especificadas mais de uma dimensão.

\begin{Shaded}
\begin{Highlighting}[]
\NormalTok{uma\_matriz[}\DecValTok{1}\NormalTok{, }\FunctionTok{c}\NormalTok{(}\StringTok{"C2"}\NormalTok{, }\StringTok{"C3"}\NormalTok{)] }\CommentTok{\#elementos na primeira linha, segunda e terceira colunas}
\end{Highlighting}
\end{Shaded}

\begin{verbatim}
## C2 C3 
##  5  9
\end{verbatim}

\begin{Shaded}
\begin{Highlighting}[]
\NormalTok{uma\_matriz[}\DecValTok{1}\NormalTok{, ] }\CommentTok{\#todos elementos da primeira linha}
\end{Highlighting}
\end{Shaded}

\begin{verbatim}
## C1 C2 C3 
##  1  5  9
\end{verbatim}

\begin{Shaded}
\begin{Highlighting}[]
\NormalTok{uma\_matriz[, }\FunctionTok{c}\NormalTok{(}\StringTok{"C2"}\NormalTok{, }\StringTok{"C3"}\NormalTok{)] }\CommentTok{\#todos elementos da segunda e terceira colunas}
\end{Highlighting}
\end{Shaded}

\begin{verbatim}
##    C2 C3
## L1  5  9
## L2  6 10
## L3  7 11
## L4  8 12
\end{verbatim}

\begin{Shaded}
\begin{Highlighting}[]
\NormalTok{uma\_matriz[, }\FunctionTok{c}\NormalTok{(}\DecValTok{2}\NormalTok{, }\DecValTok{3}\NormalTok{)] }\CommentTok{\#todos elementos da segunda e terceira colunas}
\end{Highlighting}
\end{Shaded}

\begin{verbatim}
##    C2 C3
## L1  5  9
## L2  6 10
## L3  7 11
## L4  8 12
\end{verbatim}

\subsubsection{Combinando matrizes}\label{combinando-matrizes}

Considere a seguinte matriz.

\begin{Shaded}
\begin{Highlighting}[]
\NormalTok{outra\_matriz }\OtherTok{\textless{}{-}} \FunctionTok{matrix}\NormalTok{(}
      \FunctionTok{seq}\NormalTok{(}\DecValTok{2}\NormalTok{, }\DecValTok{24}\NormalTok{, }\DecValTok{2}\NormalTok{),}
      \AttributeTok{nrow =} \DecValTok{4}\NormalTok{,}
      \AttributeTok{dimnames =} \FunctionTok{list}\NormalTok{(}
        \FunctionTok{c}\NormalTok{(}\StringTok{"L5"}\NormalTok{, }\StringTok{"L6"}\NormalTok{, }\StringTok{"L7"}\NormalTok{, }\StringTok{"L8"}\NormalTok{),}
        \FunctionTok{c}\NormalTok{(}\StringTok{"C5"}\NormalTok{, }\StringTok{"C6"}\NormalTok{, }\StringTok{"C7"}\NormalTok{)}
\NormalTok{      ))}
\end{Highlighting}
\end{Shaded}

A combinação de matrizes pode ser feita através das funções cbind e rbind, que combina matrizes por colunas e por linhas, respectivamentes.

\begin{Shaded}
\begin{Highlighting}[]
\FunctionTok{cbind}\NormalTok{(uma\_matriz, outra\_matriz)}
\end{Highlighting}
\end{Shaded}

\begin{verbatim}
##    C1 C2 C3 C5 C6 C7
## L1  1  5  9  2 10 18
## L2  2  6 10  4 12 20
## L3  3  7 11  6 14 22
## L4  4  8 12  8 16 24
\end{verbatim}

\begin{Shaded}
\begin{Highlighting}[]
\FunctionTok{rbind}\NormalTok{(uma\_matriz, outra\_matriz)}
\end{Highlighting}
\end{Shaded}

\begin{verbatim}
##    C1 C2 C3
## L1  1  5  9
## L2  2  6 10
## L3  3  7 11
## L4  4  8 12
## L5  2 10 18
## L6  4 12 20
## L7  6 14 22
## L8  8 16 24
\end{verbatim}

\subsubsection{Operações com matrizes}\label{operauxe7uxf5es-com-matrizes}

As operações básicas (+, -, *, /) funcionam de elemento a elemento em matrizes, da mesmo forma como em vetores:

\begin{Shaded}
\begin{Highlighting}[]
\NormalTok{uma\_matriz }\SpecialCharTok{+}\NormalTok{ outra\_matriz}
\end{Highlighting}
\end{Shaded}

\begin{verbatim}
##    C1 C2 C3
## L1  3 15 27
## L2  6 18 30
## L3  9 21 33
## L4 12 24 36
\end{verbatim}

\begin{Shaded}
\begin{Highlighting}[]
\NormalTok{uma\_matriz }\SpecialCharTok{*}\NormalTok{ outra\_matriz}
\end{Highlighting}
\end{Shaded}

\begin{verbatim}
##    C1  C2  C3
## L1  2  50 162
## L2  8  72 200
## L3 18  98 242
## L4 32 128 288
\end{verbatim}

Cuidado: as matrizes e vetores devem ter tamanhos compativeis!

  \bibliography{book.bib,packages.bib}

\end{document}
